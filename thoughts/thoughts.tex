% GO TO: /Users/leonlufkin/Library/texmf/tex/latex
%        to add custom packages to the path
\documentclass{article}
\usepackage{report}

\title{Receptive Field Localization}
\author{Leon Lufkin}
\date{\today}

\makeatletter
\let\Title\@title
\let\Author\@author
\let\Date\@date

\begin{document}

%%%%%%%%%%%
%% Model %%
%%%%%%%%%%%
\section{Model}
We begin with a general model that encapsulates many of the works we will discuss.
For an input $\mathbf{x}$, we want to predict a response $\mathbf{y}^*$.
Our prediction is a function $f_\theta$ of $\mathbf{x}$.
The function $f_\theta$ is parameterized by $\theta$.
We will consider models of the form
\begin{equation}
  \label{eq:general_model}
  f_\theta(\mathbf{x}) = y_\theta \left( \sum_i \alpha_\theta^{(i)} \phi_\theta^{(i)} (\mathbf{x}) \right).
\end{equation}
Here, $\mathbf{x}$ is an $L$-dimensional real vector and $\mathbf{y}^*$ is an $M$-dimensional real vector.
The $\phi_i$ are basis functions that map from $\R^L$ to $\R^K$, and the $\alpha_i$ are real scalars.
Additionally, the sum over $i$ need not be finite.
The coefficients $\alpha_i$ may depend on $\mathbf{x}$ and our parameters $\theta$, but we suppress the former dependence for notational simplicity. 
The function $y_\theta$ maps from $\R^K$ to $\R^M$ and is parameterized by $\theta$ as well.

We will show how this model encapsulates the works we are interested in understanding.

%%%%%%%%%%%%%%
%% Ingrosso %%
%%%%%%%%%%%%%%
\section{Ingrosso et al. (2022)}
The model in Ingrosso et al. (2022) sets $M = 1$, uses linear basis functions, and sets $y_\theta$ to be the mean function after applying a nonlinearity.
That is,
\begin{align}
  \sum_i \alpha_\theta^{(i)} \phi_\theta^{(i)} (\mathbf{x}) &= \Theta \mathbf{x} + b_\theta \\
  y_\theta(\mathbf{x}) &= \frac{1}{K} \bm{1}^\top \sigma(\mathbf{x}).
\end{align}
Here, $\Theta$ is a $K \times L$ matrix, $b_\theta$ is a $K$-dimensional vector, and $\sigma$ is a nonlinear function applied elementwise.



\end{document}
