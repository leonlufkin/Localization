% GO TO: /Users/leonlufkin/Library/texmf/tex/latex
%        to add custom packages to the path
\documentclass{article}
\usepackage{report}

\title{Literature Review \& Summary}
\author{Leon Lufkin}
\date{\today}

\makeatletter
\let\Title\@title
\let\Author\@author
\let\Date\@date

\begin{document}

\section{Internship Summary}
\label{sec:internship_summary}
The goal of my internship was to come up with a normative model for the emergence of the structure shown in Alessandro's paper \cite{ingrosso2022data}.
There were three phenomena: splitting into two populations of neurons, tiling, and the emergence of localized receptive fields (RFs).
The first phenomenon was relatively easy to explain, and the second is quite challenging mathematically but makes a lot of intuitive sense.
The third phenomenon, the emergence of localized RFs, is the most intriguing and the least understood.
It is especially intriguing because of its apparent connection to the development of localized, oriented edge detectors in primary visual cortex, and perhaps also place cells in hippocampus.

Here, I will explore to what extent we could explain that phenomenon using some of the analysis I did during my internship.
First, I will summarize my work on understanding localization.
Then, I will summarize existing models of visual cortex development.
Following that, I will summarize existing work on the emergence of place cells.
Finally, I will discuss whether it makes sense to try to connect the two, and if so, how we might do that.

\section{Localization}
\label{sec:localization}
The starting point for our work is the two-layer feedforward neural network:
\begin{align}
    \hat{y}(x) &= \left( \sum_{i=1}^K w_i \sigma\left(\langle w_i, x \rangle + b_i \right) \right) + b, \label{eq:two_layer_network}
\end{align}
for some nonlinearity $\sigma : \R \to \R$.
For the sake of analysis, we simplify the network to be a soft-committee machine:
\begin{align}
    \hat{y}(x) &= \frac{1}{K} \sum_{i=1}^K \sigma\left(\langle w_i, x \rangle + b_i \right). \label{eq:soft_committee_machine}
\end{align}
We will ignore details about weight initialization for now, but we typically initialize the weights $w_i$ to be i.i.d.\ from a Gaussian distribution with mean zero.

Let us consider data $(X, Y)$ from a distribution $p$ on $\R^n \times \R$ satisfying the following assumptions:
\begin{enumerate}
    \item \emph{Symmetry about 0}: $p(x) = p(-x)$ for all $x \in \R^n$.
    \item \emph{Translation invariance}: $p(x) = p(S^i x)$ for all $x \in \R^n$ and $i \in \{1, \ldots, n\}$, where $S$ is the shift operator.
    \item \emph{Localized covariance}: $\Sigma(i,j) = 0$ if $|i-j| > k$ for some $k$.
\end{enumerate}

\subsection{Theoretical Results}
We consider a single ReLU neuron for simplicity.
The gradient flow is
\begin{align}
    \tau \frac{d}{dt} w &= \frac{1}{2} \underbrace{\E_{X \mid Y=1} \left[ \mathbbm{1}(\langle w, X \rangle \geq 0) X \right]}_{\triangleq f(w)} - \left( \Sigma_0 + \Sigma_1 \right) w.
\end{align}
We analyze $f$ entrywise:
\begin{align}
    f_i(w) 
    &= \E_{X \mid Y=1} \left[ \mathbbm{1}(\langle w, X \rangle \geq 0) X_i \right] \\
    &= \E_{X_i \mid Y=1} \left[ \mathbbm{P}_{X \mid X_i, Y=1}(\langle w, X \rangle \geq 0) X_i \right] \\
    &= \E_{X_i \mid Y=1} \left[ \left( \Phi\left( \frac{\langle w, \mu_{\mid X_i} \rangle}{\sqrt{ w^\top \Sigma_{\mid X_i} w}} \right) - \frac{1}{2} \right) X_i \right] + \eps(w) \\
    &= \E_{X_i \mid Y=1} \left[ \operatorname{erf}\left( \frac{1}{\sqrt{2}} \cdot \frac{\langle w, \mu_{\mid X_i} \rangle}{\sqrt{ w^\top \Sigma_{\mid X_i} w}} \right) X_i \right] + \eps(w),
\end{align}
where $\eps(w)$ is the error term, acquired by assuming $X \mid X_i, Y=1$ is Gaussian.

Marginal concentration is not necessary (though it seems to be sufficient!), as I can construct counterexamples.
% What's really important is how $\frac{\langle w, \mu_{\mid X_i} \rangle}{\sqrt{ w^\top \Sigma_{\mid X_i} w}}$ behaves as a function of $X_i$.
% When $X$ is Gaussian, where we don't see localization, it depends linearly on $X_i$.
% When $X$ is high-gain, it is linear and constant w.r.t. $X_i$ (because of marginal concentration).
% In the intermediate-gain case, where we see localization, it depends superlinearly on $X_i$.
% I can construct examples where it is the sign function, and we still get localization.

Assuming just $\mu_{\mid X_i} \propto X_i$ (which I would consider a very reasonable assumption), we can conclude that $\mu_{\mid X_i} = \sigma_i X_i$, where $\sigma_i$ is the $i$-th row in our covariance matrix $\Sigma$ for the corresponding class (either $Y=0$ or $Y=1$).
It is harder to make statements about $\Sigma_{\mid X_i}$.
However, let us momentarily assume it has the form $\Sigma_{\mid X_i} = \Sigma - \sigma_i \sigma_i^\top$, which holds in both the Gaussian and high-gain cases (though not exactly for intermediate-gain, though this is not really an issue).
Then, we can write 
\begin{align}
    f_i(w) 
    &= \E_{X_i \mid Y=1} \left[ \operatorname{erf}\left( \frac{X_i}{\sqrt{2}} \operatorname{alg}^{-1} \left( \frac{\langle w, \sigma_i \rangle}{\sqrt{w^\top \Sigma w}} \right) \right) X_i \right] + \eps(w),
\end{align}
where $\operatorname{alg} : \R \to (-1, 1)$ is the algebraic sigmoid function, defined by $\operatorname{alg}(x) = \frac{x}{\sqrt{1+x^2}}$.
Defining $\varphi(a) = \E_{X_i \mid Y=1} [ \operatorname{erf}( X_i / \sqrt{2} \operatorname{alg}^{-1} ( a ) ) X_i ]$, we can simply write
\begin{align}
    f_i(w) 
    &= \varphi \left( \frac{\langle w, \sigma_i \rangle}{\sqrt{w^\top \Sigma w}} \right) + \eps(w).
\end{align}
Note that $\varphi(a) = \sqrt{\frac{2}{\pi}} a$ when $X$ is Gaussian.
In fact, for small $a$, we can always write $\varphi(a) \approx \sqrt{\frac{2}{\pi}} a$ (assuming unit marginal variance).
If $X_i$ is bimodal, then $\varphi(a) > \sqrt{\frac{2}{\pi}} a$ for larger $a$.
The opposite seems to be true if it has smaller tails than the Gaussian.
This is based on testing a bunch of examples, though this makes intuitive sense as well.
\emph{I will think about how to formalize this intuition.}

Using the above notation, it is natural to consider the large $n$ limit and make our system of coupled ODEs a PDE.
Treating $w$ as a function of position $x$ and time $t$, and ignoring $\eps$, we can write
\begin{align*}
    \tau \frac{\partial}{\partial t} w(x, t) &= \frac{1}{2} \varphi\left( \frac{ \Sigma_1 \star w }{ \langle \Sigma_1 \star w, w \rangle } \right) - \left( \Sigma_0 + \Sigma_1 \right) \star w,
\end{align*}
where the convolutions are performed over the spatial domain.
We can recover the same general results from this PDE: when $\varphi(a) \propto a$, i.e. the Gaussian setting, we have an oscillatory steady state, while when $\varphi(a)$ is superlinear, we have a localized steady state, though the latter I can only show empirically.
This PDE almost certainly does not have an explicit solution.
However, it may be possible to show that the steady state is sparse when $\varphi(a)$ is superlinear.
I have been thinking about this for a while, but I have not made much progress.
\emph{I am continuing to think about it.}

\subsubsection{Removing the normal approximation}
Let us return to our definition of $f_i(w)$.
To get an explicit form of the expectation, we previously assumed $X \mid X_i$ is Gaussian.
Now, let us try to dispense with this assumption by using tools like Chebyshev's inequality instead.

We focus on the term $\PR_{X \mid X_i, Y=1}( \langle w, X \rangle \geq 0 )$.
Let $\mu_i \triangleq \E_{X \mid X_i, Y=1}[ \langle w, X \rangle ]$ and $\sigma_i^2 \triangleq \V_{X \mid X_i, Y=1}[ \langle w, X \rangle ]$.
Note that $\mu_i = \langle w, \mu_{\mid X_i} \rangle$ and $\sigma_i^2 = w^\top \Sigma_{\mid X_i} w$.
By Chebyshev's inequality, we have for all positive $k$
\begin{align*}
    \PR_{X \mid X_i, Y=1}( | \langle w, X \rangle - \mu_i | \geq k \sigma_i ) \leq \frac{1}{k^2}
    \iff
    \PR_{X \mid X_i, Y=1}( | \langle w, X \rangle - \mu_i | > k \sigma_i ) \geq 1 - \frac{1}{k^2}.
\end{align*}
Note that $| \langle w, X \rangle - \mu_i | < k \sigma_i \implies -(\langle w, X \rangle - \mu_i) < k \sigma_i \iff \langle w, X \rangle > \mu_i - k \sigma_i$.
This implies that
\begin{align*}
    \PR_{X \mid X_i, Y=1}( \langle w, X \rangle > \mu_i - k \sigma_i )
    &\geq \PR_{X \mid X_i, Y=1}( | \langle w, X \rangle - \mu_i | > k \sigma_i )
    \geq 1 - \frac{1}{k^2}.
\end{align*}
Let us assume that $\mu_i > 0$.
Then, picking $k = \frac{\mu_i}{\sigma_i}$, we get
\begin{align*}
    \PR_{X \mid X_i, Y=1}( \langle w, X \rangle > 0 )
    &\geq 1 - \frac{\sigma_i^2}{\mu_i^2} \\
    &= 1 - \frac{w^\top \Sigma_{\mid X_i} w}{\langle w, \mu_{\mid X_i} \rangle^2}
    = 1 - \frac{w^\top \Sigma_1 w - (\langle w, \sigma_i \rangle)^2}{X_i^2 \langle w, \sigma_i \rangle^2}
    = 1 - \frac{1}{X_i^2} \left( \frac{w^\top \Sigma_1 w}{(\langle w, \sigma_i \rangle)^2} - 1 \right) \\
    &= 1 - \frac{1}{X_i^2} \left( \frac{1}{a^2} - 1 \right),
\end{align*}
where $a = \frac{\langle w, \sigma_i \rangle}{\sqrt{w^\top \Sigma_1 w}}$.

Recall that Chebyshev's bound is only useful when $k > 1$.
That is, $$ \mu_i > \sigma_i \iff X_i \langle w, \sigma_i \rangle > \sqrt{w^\top \Sigma_1 w - (\langle w, \sigma_i \rangle)^2} \iff \frac{1}{\sqrt{1/a^2 - 1}} = \frac{ \langle w, \sigma_i \rangle }{ \sqrt{w^\top \Sigma_1 w - (\langle w, \sigma_i \rangle)^2} } > \frac{1}{X_i}. $$
This can be further rewritten as $$ X_i > \sqrt{1/a^2 - 1} \iff X_i^2 + 1 > 1/a^2 \iff a = \frac{\langle w, \sigma_i \rangle}{\sqrt{w^\top \Sigma_1 w}} > \frac{1}{\sqrt{X_i^2 + 1}}. $$
So, it's more likely to be useful for larger values of $X_i$, or where $a$ is large ($w$ looks localized).

Let us use this to bound $f_i(w)$.
\begin{align*}
    f_i(w)
    &= \E_{X_i \mid Y=1} \left[ \PR_{X \mid X_i, Y=1}( \langle w, X \rangle > 0 ) X_i \right] \\
    &= \E_{X_i > 0 \mid Y=1} \left[ \PR_{X \mid X_i, Y=1}( \langle w, X \rangle > 0 ) X_i \right] + \E_{X_i < 0 \mid Y=1} \left[ \PR_{X \mid X_i, Y=1}( \langle w, X \rangle > 0 ) X_i \right] \\
    &= \E_{X_i > 0 \mid Y=1} \left[ \PR_{X \mid X_i, Y=1}( \langle w, X \rangle > 0 ) X_i \right] - \E_{X_i > 0 \mid Y=1} \left[ \PR_{X \mid -X_i, Y=1}( \langle w, X \rangle > 0 ) X_i \right] \\
    &= \E_{X_i > 0 \mid Y=1} \left[ \PR_{X \mid X_i, Y=1}( \langle w, X \rangle > 0 ) X_i \right] - \E_{X_i > 0 \mid Y=1} \left[ \PR_{X \mid X_i, Y=1}( -\langle w, X \rangle > 0 ) X_i \right] \\
    &= \E_{X_i > 0 \mid Y=1} \left[ \left( \PR_{X \mid X_i, Y=1}( \langle w, X \rangle > 0 ) - \PR_{X \mid X_i, Y=1}( -\langle w, X \rangle > 0 ) \right) X_i \right] \\
    &= \E_{X_i > 0 \mid Y=1} \left[ \left( 2 \PR_{X \mid X_i, Y=1}( \langle w, X \rangle > 0 ) - 1 \right) X_i \right] \\
    &= 2 \E_{X_i > 0 \mid Y=1} \left[ \PR_{X \mid X_i, Y=1}( \langle w, X \rangle > 0 ) X_i \right] - \E_{X_i \mid Y=1} \left[ |X_i| \right].
\end{align*}
If $\langle w, \sigma_i \rangle \geq 0$, then $\langle w, \mu_{\mid X_i} \rangle \geq 0$ for $X_i > 0$.
Thus, $\mu_i > 0$ and we can write
\begin{align*}
    f_i(w)
    &\geq 2 \E_{X_i > 0 \mid Y=1} \left[ \left( 1 - \frac{1}{X_i^2} \left( \frac{1}{a^2} - 1 \right) \right) X_i \right] - \E_{X_i \mid Y=1} \left[ |X_i| \right] \\
    &= \E_{X_i \mid Y=1} \left[ |X_i| \right] - 2 \left( \frac{1}{a^2} - 1 \right) \E_{X_i \mid Y=1} \left[ \frac{1}{|X_i|} \right] \\
\end{align*}


\subsection{Empirical Results}
I was able to show during my internship that a model trained on elliptical data will not localize.
I showed this theoretically for ReLU activation, and confirmed it empirically in \cref{fig:ellipse_rf}.
(It looks jagged because of a numerical issue from how I construct samples from arbitrary elliptical distributions.
Using a distribution with existing sampling algorithms, like the multivariate $t$, works better.)
This was interesting because it seemed to run contrary to the standard argument that kurtosis drives localization (as mentioned in \cite{brito2016nonlinear} and \cite{ingrosso2022data}).
However, it did not technically disprove any previous results, since those examples could be explained by the fact that the marginal distributions were concentrated.

\begin{figure}[!ht]
    \centering
    \includegraphics[width=0.5\textwidth]{elliptical_RF.png}
    \caption{The RF of a model trained on elliptical data.}
    \label{fig:ellipse_rf}
\end{figure}

Elliptical distributions are very flexible.
The only constraint they impose is radial symmetry.
This leads one to wonder whether violating radial symmetry may be sufficient for localization.

In general, we can write a density $p(x)$ in terms of $p(r, \theta)$, where $\theta$ is an angle and $r$ is the norm of $x$ under some (possibly weighted) norm.
Let us consider distributions that can be factorized as $p(r, \theta) = p(r) p(\theta)$.
This contains elliptical distributions, since we assume that $p(\theta)$ is constant and $p(r)$ parameterizes an elliptical, which is easy to see if we define $r = \sqrt{x^\top \Sigma^{-1} x}$ (a weighted norm).

Let's consider a setting where $p(\theta)$ disfavors some angles around $\theta_0$ by having $p(\theta) = 0$ for $\theta \in (\theta_0-\eps,\theta_0+\eps)$.
% What if we tweak $p(\theta)$ to disfavor some point over others?
% (I need to clarify exactly what this means.)
Recall that we need to make it satisfy sign- and translation-invariance.
So, we get that it disfavors $2n$ neighborhoods spaced evenly along both sides of the $n$ axis lines, up to some rotation.
Because we have factorized the density as $p(r) p(\theta)$, this immediately implies the distribution has concentrated marginals, up to some rotation.
This means that, in rotated space, $w$ will be localized.

This is not rigorous, and I have not thoroughly tested this, but I have run some experiments to suggest this may be true.
Below, I show the RF for a model trained on an elliptical distribution where I threw out points that were aligned along an axis.
[\emph{I'm still working on fixing this figure, since the numerical error seems to become more of a problem here. But I was able to get the RF from \cref{fig:ellipse_rf} to localize by just throwing out some samples like how I mentioned above.}]

Of course, this is only one way to break radial symmetry, but it does suggest that we are somewhat close to having a sharp distinction between distributions that will localize and those that won't.

% The key questions:
% \begin{enumerate}
%     \item Can this approach help us identify a crisp distinction between which distributions will yield localization and which won't? (That is, can we establish clear positive and negative results?)
%     \item Could we even test those predictions? (We need to do this because we make an approximation we don't really understand in the analysis.)
% \end{enumerate}

% Can we usually write
% \begin{align*}
%     \frac{\langle w, \mu_{\mid X_i} \rangle}{\sqrt{ w^\top \Sigma_{\mid X_i} w}} &= \frac{\langle w, \sigma \rangle}{\sqrt{ w^\top (\Sigma - \sigma \sigma^\top) w}} f(X_i),
% \end{align*}
% for some $f$?

% What does it mean for $\Sigma_{\mid X_i}$ to depend on the value of $X_i$, as opposed to just the fact that we are conditioning on position $i$?






\section{V1 RFs}
\label{sec:v1_rfs}
Primary visual cortex has been studied extensively.
A variety of works have shown that neurons in V1 have localized RFs that are oriented and bandpass, and thus modeled well by Gabor functions.
A rich history of work has searched for computational mechanisms that can reproduce the variety of RFs observed in V1.
The seminal work in this area is \cite{olshausen1996emergence}, who showed a feedforward neural network with continuous activation trained on (whitened) natural images to minimize $L^2$ reconstruction error using gradient descent, plus a sparsity penalty, could learn Gabor-like filters.
The sparse coding constraint was inspired by the fact that V1 neurons are sparse.
While this constraint is capable of producing Gabor-like filters, it is not clear how V1 would impose such a constraint, nor why it would need to.

% \textbf{Talk about ICA.}
Another important line of work is independent component analysis (ICA).
ICA works by finding a linear transformation of the data such that the components are as independent as possible, as opposed to PCA, which seeks to minimize correlation.
ICA suggests the emergence of orientation selectivity is the result of reducing redundancy among higher-order statistics.
However, \cite{eichhorn2009natural} suggests this is unlikely to be the case, since orientation selectivity does not actually lower redundancy much.

Another important work I came across is \cite{brito2016nonlinear}.
They show that nonlinear Hebbian learning can generalize foundational normative models of localized RFs, like sparse coding and ICA, as well as bottom-up approaches, such as the Bienenstock-Cooper-Munro model of synaptic plasticity.
They argue and provide evidence that this explains why all these models yield qualitative similar results, and come up with a heuristic based on kurtosis for whether a given nonlinearity will yield localized RFs.
This is a substantial generalization of Andrew's work \cite{saxe2011modeling}, insofar as he also showed that a variety of algorithms can qualitatively reproduce the RFs seen throughout sensory cortex.

The mathematics in \cite{brito2016nonlinear} is a bit hand-wavy, but it does seem to be capable of explaining a lot of the variation seen in whether an algorithm will yield localized RFs when trained on natural images.
However, their work clearly suggests that without a sparsity penalty, a model trained with vanilla gradient descent will not localize (see Fig. 3d).
\cite{ingrosso2022data} somewhat disprove this with their synthetic data model, which \cite{pellegrini2022neural} shows can hold when training on ImageNet32, although they needed to use pruning to extract the embedded localized RFs.
\cite{redman2023sparsity} shows that adding an L1 penalty encourages localization (I feel like this result is a bit obvious, though, since an L1 penalty encourages sparsity, which is likely localized).

One of the most thorough ``bottom-up'' models of V1 development I've seen is \cite{zylberberg2011sparse}.
Here, they show that a network of leaky-integrate-and-fire neurons updated using Oja's rule can capture the variety of filters seen in V1.
Importantly, similar to \cite{brito2016nonlinear}, they show that their learning rule can be interpreted as a constrained optimization problem, where the constraint is that the neurons' firings are sparse.
An important point they make is that their model uses a \emph{local} learning rule, whereas the original sparse coding model used a \emph{global} learning rule, which is not as biologically plausible.
Our setting uses a global learning rule, which is a potential weakness, though I don't think this matters for the questions we are asking.

\subsection*{What is left to be explained?}
% The above works explain a lot about V1 RFs, but there are still some open questions.
To the best of my knowledge, no work explains \emph{why V1 RFs have a Gabor/sinusoidal structure}.
The usual intuition is that it's because Gabor wavelets are optimal for representing images with minimal space-frequency uncertainty.
Our work suggests that this may be driven by the structure of the covariance of whitened natural images and the separation of their marginal distributions.
However, we need to be able to account for the variety of orientations and phases of V1 RFs.
Extending Alessandro's toy model can account for the variety of phases, but it's not clear how to account for the variety of orientations.

In this vein, another unanswered question is \emph{why V1 RFs are localized}, or perhaps more precisely, \emph{what about natural images drives localization?}
It is easy to construct a network that is not localized yet performs equally well (at least in the simple model of \cite{ingrosso2022data}).
Our work establishes some sufficient conditions for localization, though it is not clear whether these are necessary, as well some negative results that seem to run contrary to the standard argument that kurtosis drives localization (as mentioned in \cite{brito2016nonlinear} and \cite{ingrosso2022data}).
My current impression is that this might be the best place for us to make a contribution given the results we have, though we definitely still have more technical work to do to make this rigorous.

\subsection*{Where would we fit in?}
As I understand it, Andrew's original idea was to address the first question above: to use our work to explain why V1 RFs have a Gabor/sinusoidal structure.
To make this connection, we need to move away from the toy model and towards natural images, and also towards an unsupervised learning setting.
We would also probably need to show analytically how the steady state of the localizing ODE is related to the covariance of the data, since it's not obvious from staring at the equation that the steady state is localized with bandwidth close to the covariance.

If we were instead to focus on the second question above, we would have to connect our toy setting to an unsupervised setting based on natural images that can capture a variety of orientations and off-phase RFs.
We identified this challenge when we met with Andrew in mid-December, but I still have not been able to come up with a good solution.
It seems that in order to get this to work, we would have to either tweak with the data model or the learning rule so much that my existing analysis would not be relevant.
There is certainly still an empirical approach to be taken here by, say, constructing a dataset of images and slowly tweaking the covariance and marginals to see how that affects the RFs, and hopefully showing it's consistent with our analysis in the toy setting.
However, I haven't had a chance to explore this yet.

Additionally, to make the jump from artificial neural networks to biological ones, we would probably need to connect our vanilla gradient descent setting to a biologically plausible one as discussed in \cite{brito2016nonlinear}.

% Additionally, there is the overarching concern that our model uses \emph{vanilla} gradient descent, while biologically-plausible models effectively always incorporate some form of regularization (as discussed in \cite{brito2016nonlinear}).
% the idea was to propose that vanilla gradient descent is sufficient to produce Gabor-like RFs in response to natural images.
% Specifically, we would be trying to characterize \emph{which aspects of natural image statistics} are necessary and sufficient for learning localized Gabor-like RFs.



% To my understanding, this question has never been asked precisely. % (perhaps empirically, but certainly not analytically).
% [Brite \& Gerstner (2016)] argue that Nonlinear Hebbian Learning is the underlying algorithm driving RF formation across sensory modalities.
% (They show sparse coding and (maybe?) vanilla MSE gradient descent are special cases of this, and come up with a metric for whether a given nonlinearity will localize. The motivation is not super analytically precise, relying on intuitions about kurtosis, but it seems to kind of work. Cool stuff!)
% This implies, as they note, that it's the underlying statistics of the sensory input that drive the nature of the RFs.
% However, they are not addressing whether an RF is localized, but rather what the already-localized RF looks like.


% do not characterize what those statistics are.

% It is typically asserted 

% Is the idea, then, to propose vanilla gradient descent as sufficient, but only for specific distributions?

\subsection*{Summary}
In summary, there are three main concerns I have with how to position our current work:
\begin{enumerate}
    \item We presently only have sufficient conditions for localization (marginal concentration) and a negative result (elliptical distributions do not localize).
    There might be a way to sharpen this distinction, but I haven't made this precise yet.
    \item I don't think we can produce the full variety of RFs observed in V1 using \emph{this data model}.
    Specifically, yielding both cosine- and sine-phase RFs, and the variety of orientations.
    Based on some preliminary experiments, it does seem possible to achieve a variety of spatial frequencies by having the data be a mixture of many frequencies, instead of just two.
    However, it is not clear how to achieve a variety of orientations.
    \item Our setting is not biologically plausible. 
    First we use a synthetic data model, and it is not clear that we can extend this to a setting where we use natural images without additional regularization (see \cite{redman2023sparsity}).
    To do this, we'd also need to make the jump from supervised to unsupervised learning.
    This might be possible without our current data model, but it's not clear how that this should work in a more general setting.
    Furthermore, in the setting of natural images, it's not clear how the analysis we have so far would be relevant.
\end{enumerate}

I am continuing to work on item 1.
However, I feel that if we could come up with tighter positive and negative results for localization, that might even be best as a standalone paper, or at least to have that be the main contribution of the paper, rather than try to establish some connection back to V1.

Andrew floated directional derivatives as a way to get a variety of orientations among RFs to address item 2.
However, I am not really sure how to implement this, or what it would mean to a neuroscientist.
\emph{It might be best to discuss this with Andrew again.}

Regarding relevance to a neuroscientist, we still have the issue of item 3.
To make any neuroscientist find meaning in our work, we need to move away from the toy model and towards a more biologically plausible, as seems to be standard in many of the works I've read.



\section{Place Cells}
\label{sec:place_cells}
While looking for related work, Erin and I came across \cite{harsh2020place}, which is cited in \cite{ingrosso2022data}.
This paper uses a restricted Boltzmann machine (RBM) to model hippocampal place cells.
They sample data from an Ising model to train the RBM.
In particular, they show analytically that the RBM learns localized RFs.
This is the only work I've come across that explicitly models place cells, and I think this could be a more relevant and exciting application of our work.
It seems that the question of why place cells emerge is still open, and I think our work could be a step in the right direction.

\cite{harsh2020place} just focuses on the Ising model and doesn't try to make general claims about which distributions will lead to place-cell emergence.
They connect their findings to previous work on ``retarded learning,'' a general phenomenon in which a minimum of data is needed for a phenomenon (in this case, place cells) to emerge.
This doesn't seem like a direction we can readily tackle, since we've been operating in the full-batch setting.

\cite{harsh2020place} recalls Gardner's theory of optimal learning for a single-layer perceptron \cite{gardner1988space}, which predicts that data drawn from a translation-invariant distribution will lead to the emergence of localized RFs.
At a first glance this seems to make a (false) prediction, but upon closer reading it actually seems to just be making a prediction about the Ising model.
Nevertheless, \cite{harsh2020place} make it sound like translation-invariant data is a sufficient condition for place-cell emergence, which is not true.
It seems like our work, which draws a sharper distinction, might be a useful contribution here.

\subsection*{Summary}
\emph{I have not found much other work on modeling place cells, but I will keep looking.}
This seems like an underexplored area, and I think it could be a good application of our work.
Our toy model is much closer to existing work in this space than for work on V1.
Additionally, localization and place-cell emergence are basically the same thing, whereas V1 RFs have the additional complexity of orientation and phase selectivity that I have not been able to address yet.


%% Works Cited
% \newpage
\bibliographystyle{alpha}
\bibliography{refs}


\end{document}
